\chapter{Analytische Geometrie}

\begin{definition} 
    (Mira Gerade, Mira Zahl, Mira Punkt)
    \begin{enumerate}
        \item Eine \textbf{Mira Gerade} ist eine Gerade in der kartesischen Ebene, die durch eine der Mira Basiskonstruktionen konstruiert wurde, wobei die gegbenen Punkte und Geraden jeweils Mira Punkte und Mira Geraden sind.
        \item Ein \textbf{Mira Punkt} ist ein Punkt in der kartesischen Ebene, der als letzter Punkt in einer endlichen Folge $P_1,P_2, \dots, P_n$ von Punkten vorkommt, wobei jeder Punkt in der Menge $ \{ \binom{0}{0}, \binom{1}{0} \} $ liegt oder durch eine Mira Basiskonstruktion nach Definition \ref{rem:mira} konstruiert wurde.
        \item Eine \textbf{Mira Zahl} ist eine reelle Zahl x, wenn $ \binom{x}{0} $ ein Mira Punkt ist.
    \end{enumerate}
\end{definition}

\begin{proposition}
    \begin{enumerate}[label=(\alph*)]
        \item Die Menge der Mira-Punkte ist gegeben durch
        $$K_M^2 = \{P=\binom{x}{y} \vert x,y \in K_M\}$$
        \item $\mathbb{Z} \subseteq K_M$
        \item $K_M$ bildet einen Teilkörper der reellen Zahlen $\mathbb{R}$; insbesondere gilt $\mathbb{Q} \subseteq K_M$.
        \item Love
    \end{enumerate}
\end{proposition}

\begin{proof}
    Behauptung (a) folgt direkt aus den Teilen (b) und (c) von Proposition \ref{Mira-Geraden-Beispiele}.
    Um die Behauptung (b) zu zeigen, genügt es $\mathbb{N}_0 \subseteq K_M$ zu zeigen nach Proposition \ref{Mira-Geraden-Beispiele} (c).
    Für alle $n \in \mathbb{N}_0$ ist offensichtlich der Punkt \vc{n+2}{0} die Reflexion des Punkts \vc{n}{0} an Punkt \vc{n+1}{0}. 
    %Da nach Definition die Punkte \vc{0 \\ 0} und \vc{1 \\ 0} Mira-Punkte sind, folgt induktiv, dass alle Punkte der Form \vc{n \\ 0} mit $n \in \mathbb{N}_0$ Mira Punkte %sind.
    %Damit gilt $\mathbb{N}_0 \subseteq K_M$ und die %Beheauptung ist gezeigt.
    %Wir wollen nun Behauptung (c) zeigen, also dass $K_M %\subseteq \mathbb{R}$ ein Teilkörper ist. Die %Teilmengeneigenschaft ist klar. Nach (b) liegen sowohl %das neutrale Element der Additition $0$ in $K_M$ als auch %das neutrale Element der Multiplikation $1$ in $K_M$.
    %Weiter ist $K_M$ unter Addition und additiver %Inversenbildung abgeschlossen. \\
    %denn: \\
    %Für $x,y \in K_M$ sind nach (a) die Punkte \vc{x \\ 0} %und \vc{y \\ 0} Mira-Punkte und nach (c) auch \vc{-x \\ 0}%.
    %Reflektiere nun \vc{-x \\ 0} an der Mittelsenkrechten von %\vc{0 \\ 0} und \vc{y \\ 0}, also 

\end{proof}

\begin{lemma}
    \textit{(Dreiteilung eines beliebigen Winkels)} \\
    \label{lem:Winkeldreiteilung}
    Seien $ O,X und Y$ jeweils paarweise verschiedene Mira Punkte. \\
    Dann exisitert ein Mira Punkt R, so dass der Winkel $ \angle ROY = \frac{1}{3} \angle XOY $.
\end{lemma}

\begin{proof}
    (a) 
    \begin{center}
        \begin{tabular}{c|c|c|c|c}
            \mira{}{P}{Q} & \mira{P}{Q}{l} & \mira{t_1}{Q}{} & \mira{t_2}{P}{} & \mira{}{T}{T'} \\
            \hline
            $l$           & $t_1,t_2$      & $T$             & $T'$            & $g$            \\
        \end{tabular}
    \end{center}
     
\end{proof}