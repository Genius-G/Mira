\chapter{Einführung}

Die Konstruktion von geometrischen Figuren ist uns allen aus der Schule bekannt.
Zumeist wurden diese durch den Einsatz von Zirkel und Lineal eingeführt.
Dass es dabei viele weitere Methoden oder Konstruktionswerkzeuge neben diesen zur Verfügung stehen ist eher unbekannt.
Wir werden uns in dieser Arbeit um die Einführung und Analyse eines Mira bemühen. 
Dieser ist ein aufrechtstehender halbtransparenter Spiegel aus einfachen rot eingefärbten Plastik.
So lassen sich virtuelle Bilder, also Reflxionen an der Stelle wo sie zu sein scheinen sichtbar machen.
In Kapitel 2 widmen wir uns der genauen Definition eines Mira im mathematischen geometrischen Sinn und definieren was eine Mira Zahl ist.
In Kapitel 3 wenden wir uns klassischen Konstruktionsproblemen zu und analysieren welche durch den Einsatz eines Mira möglich sind.
In Kapitel 4 werden wir die Ergebnisse aus den vorigen Kapiteln denen aus anderen Arbeiten zu weiteren Konstruktionswerkzeugen gegenüber stellen.