\documentclass{article}
\usepackage[utf8]{inputenc}
\usepackage[german]{babel}
\usepackage{amsthm}


\newtheorem{theorem}{Theorem}[section]
\newtheorem{corollary}{Corollary}[theorem]
\newtheorem{lemma}[theorem]{Lemma}
\newtheorem*{remark}{Remark}

\title{Geometrische Konstruktionen mit einem Mira 2}
\author{Silas G}
\date{June 2021}

\begin{document}

\maketitle

\section{Einführung}

Die Konstruktion von geometrischen Figuren ist uns allen aus der Schule bekannt.
Zumeist wurden diese durch den Einsatz von Zirkel und Lineal eingeführt.
Dass es dabei viele weitere Methoden oder Konstruktionswerkzeuge neben diesen zur Verfügung stehen ist eher unbekannt.
Wir werden uns in dieser Arbeit um die Einführung und Analyse eines Mira bemühen. 
Dieser ist ein aufrechtstehender halbtransparenter Spiegel aus einfachen rot eingefärbten Plastik.
So lassen sich virtuelle Bilder, also Reflxionen an der Stelle wo sie zu sein scheinen sichtbar machen.
In Kapitel 2 widmen wir uns der genauen Definition eines Mira im mathematischen geometrischen Sinn und definieren was eine Mira Zahl ist.
In Kapitel 3 wenden wir uns klassischen Konstruktionsproblemen zu und analysieren welche durch den Einsatz eines Mira möglich sind.
In Kapitel 4 werden wir die Ergebnisse aus den vorigen Kapiteln denen aus anderen Arbeiten zu weiteren Konstruktionswerkzeugen gegenüber stellen.

\section{Geometrie mit einem Mira}

In diesem Abschnitt werden wir Prunkte die mit einem Mira konstruierbar sind studieren und den Zahlenkörper aller Mira Zahlen bestimmen.
Die Notation und Grundlagen können gerne in Grundlagen der ebenen Geometrie nachgelesen werden.

\begin{remark}
    Mit dem Zirkel und Lineal gibt es drei Arten neue Punkte zu konstruieren.
    \begin{enumerate}
        \item Bestimme den Schnittpunkt zweier nicht paralleler Geraden
        \item Bestimme die Schnittpunkte zweier Kreise
        \item Bestimme die Schnittpunkte eines Kreises mit einer Geraden
    \end{enumerate}
\end{remark}

\end{document}
